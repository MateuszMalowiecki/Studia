\documentclass[10pt, a4paper]{article}

\usepackage[english]{babel}
\usepackage{polski}
\usepackage[utf8]{inputenc}
\usepackage{graphicx}
\usepackage{amsmath}
\title{Rozwiązania zadań 15 z listy nr. 5 i 5 z listy nr.6 z przedmiotu "Rachunek prawdopodobieństwa dla informatyków"}
\date{24 stycznia 2022}
\author{Mateusz Małowiecki}

\begin{document}
\maketitle
\section*{Zadanie 15}
\subsection*{Treść}
Przed głosowaniem nad impeachmentem prezydenta pewnego państwa, przeprowadzono sondaż. W ramach sondażu spytano n osób o to czy popierają usunięcie prezydenta. Zakładamy, że każda z pytanych osób została losowo i niezależnie od pozostałych wybrana ze zbioru osób uprawnionych do głosowania (w szczególności nie wykluczamy powtórzeń) oraz odpowiedziała TAK /NIE. Interesuje nas oszacowanie poparcia wniosku. Oszacuj jak duże powinno być n, żeby prawdopodobieństwo popełnienia błędu względnego większego niż 1\% było mniejsze 
niż 5\%.
\subsection*{Rozwiązanie}
Niech:
\begin{itemize}
\item $m$ - liczba mieszkańców państwa
\item $k$ - liczba osób, które zamierzają odpowiedzieć "TAK"
\item $X_n$ - zmienna losowa, mówiąca ilu mieszkańców odpowiedziało "TAK" spośród $n$ wybranych
\end{itemize}
Oczywiście:
\begin{equation}
X_n \sim B\Big(n, \frac{k}{m}\Big)
\end{equation}
więc:
\begin{equation}
E(X_n) = \frac{n*k}{m}
\end{equation}
Zauważmy teraz że błąd względny dany jest wzorem:
\begin{equation}
\delta_x=\frac{|x-x_0|}{|x_0|}
\end{equation}
gdzie
\begin{itemize}
\item $x$ - wartość zmierzona
\item $x_0$ - dokładna wartość
\end{itemize}
W naszym przypadku $x = \frac{X_n}{n}$ i $x_0 = \frac{k}{m}$. Zatem
\begin{equation}
\delta_x=\frac{|\frac{X_n}{n} - \frac{k}{m}|}{|\frac{k}{m}|} = \frac{|m|}{|n*k|} * \Big|X_n - \frac{k*n}{m}\Big|
\end{equation}
Korzystając z (2), możemy zauważyć, że
\begin{equation}
\delta_x = \frac{1}{E(X_n)} * |X_n - E(X_n)|
\end{equation}
Zatem
\begin{equation}
P(\delta_x \geq 0.01) = P(\frac{1}{E(X_n)} * |X_n - E(X_n)| \geq 0.01) = P(|X_n - E(X_n)| \geq 0.01 * E(X_n))
\end{equation}
Skorzystamy teraz z nierówności Chernoffa:
\begin{equation}
P(|Y - E(Y)| \geq \delta * E(Y)) \leq 2*e^{-\frac{\delta^2 * E(Y)}{3}}
\end{equation}
A zatem:
\begin{equation}
P(|X_n - E(X_n)| \geq 0.01 * E(X_n)) \leq 2*e^{-\frac{0.01^2 * E(X_n)}{3}}
\end{equation}
Zatem wystarczy znaleźć $n$ dla którego
\begin{equation}
2*e^{-\frac{0.01^2 * E(X_n)}{3}} < 0.05
\end{equation}
Po prostych przekształceniach otrzymamy:
\begin{equation}
n > 30000 * ln(4) * \frac{m}{k} \approx 41588 * \frac{m}{k}
\end{equation}
\section*{Zadanie 5}
\subsection*{Treść}
Wynik pomiaru opóźnienia w transmisji na pojedynczym odcinku sieci
komputerowej obarczony jest błędem systematycznych 0.5 i błędem losowym będącym zmienną losową o średniej 0 i wariancji 1. Dokonano pomiaru sieci składającej się ze 100 odcinków (pomiary były niezależne od siebie). Niech $Y$ będzie błędem całkowitym popełnionych przy badaniu całej sieci. Oszacować prawdopodobieństwo tego, że: \\
a) $Y < 75$ \\
b) Wynik pomiaru nie przekracza rzeczywistej wartości mierzonego opóźnienia na trasie 100 odcinków. \\
\subsection*{Rozwiązanie}
Niech $Y_i$ - opóźnienie na i-tym odcinku. Oczywiście:
\begin{equation}
Y=\sum_{i=1}^n Y_i
\end{equation} 
Zauważmy, że 
\begin{equation}
Y_i = Z_i + 0.5
\end{equation}
dla pewnych zmiennych losowych $Z_i$ takich, że $E(Z_i) = 0$ i $V(Z_i) = 1$. Zatem $E(Y_i) = 0.5$ i $V(Y_i)=1$. W celu obliczenia prawdopodobieństw skorzystamy z centralnego twierdzenia granicznego. A zatem:
\begin{equation}
U=\frac{Y - 100 *\frac{1}{2}}{\sqrt{100 * 1}} \sim N(0, 1)
\end{equation}
\subsubsection*{podpunkt a}
Chcemy oszacować $P(Y < 75)$, a zatem:
\begin{equation}
P(Y < 75) = P \Big(\frac{Y - 100 *\frac{1}{2}}{\sqrt{100 * 1}} < \frac{75 - 100 *\frac{1}{2}}{\sqrt{100 * 1}}\Big) = P (U < 2,5) \approx \Phi(2,5) = 0,993
\end{equation}
\subsubsection*{podpunkt b}
Ponieważ rzeczywista wartość opóźnienia na trasie 100 odcinków wynosi 50, więc:
\begin{equation}
P(Y < 50) = P \Big(\frac{Y - 100 *\frac{1}{2}}{\sqrt{50 * 1}} < \frac{75 - 100 *\frac{1}{2}}{\sqrt{100 * 1}}\Big) = P (U < 0) \approx \Phi(0) = 0,5
\end{equation}
\end{document}