\documentclass[10pt, a4paper]{article}

\usepackage[english]{babel}
\usepackage{polski}
\usepackage[utf8]{inputenc}
\usepackage{graphicx}
\title{Streszczenie stron 20-22(fragment sekcji "Scaling techniques") z pierwszego rozdziału książki "Distributed systems"}
\author{Mateusz Małowiecki}

\begin{document}
\maketitle
\section*{Wstęp}
Sekcja "scaling techniques" zaczyna się od krótkiego wstępu. We wstępie tym autor na początku zastanawia się, w jaki sposób można rozwiązać problemy związane ze skalowalnością. Autor rozważa problem skalowalności jako problem wydajnościowy i podaje jako rozwiązanie technikę zwaną "skalowanie w górę" (ang. scaling up - zwiększenie pojemności zasobów systemowych). Następnie autor rozważa problem "skalowania w poziomie" (ang. scaling out - rozszerzanie systemu rozproszonego poprzez wdrażanie większej liczby maszyn) i podaje trzy techniki na rozwiązanie tego problemu: ukrywanie opóźnień komunikacji(ang. hiding communication latencies), dystrybucja pracy(ang. distribution of work) i replikacja(ang. replication). My opiszemy w tym streszczeniu dwie pierwsze techniki(trzecią technikę opisze następna osoba z systemu zapisów).
\section*{Ukrywanie opóźnień komunikacji}
Na początku tej sekcji, autor podaje podstawową ideę techniki "ukrywania opóźnień komunikacji", która brzmi: "staraj się unikać czekania na odpowiedzi na żądania usług zdalnych". Następnie autor podaje w jaki sposób można wykorzystać asynchroniczną komunikację do unikania czekania na odpowiedź. Opisowi temu autor poświęca jeden akapit. Potem autor zauważa, że nie wszystkie systemy mogą wykorzystywać asynchroniczną komunikację i jako przykład podaje aplikacje interaktywne. Podaje też rozwiązanie, które jest lepsze w tym przypadku: ograniczenie ogólnej komunikacji, oraz przypadek wykorzystania tego rozwiązania: dostęp do bazy danych za pomocą formularza. Na koniec tej sekcji, autor opisuje dwa sposoby na wypełnienie formularza: 
\begin{itemize}
\item wysyłanie oddzielnej wiadomości dla każdego pola 
\item wysłanie kodu do wypełnienia formularza
\end{itemize}
Sposoby te zostały przedstawione na rysunkach 1.4(a) oraz 1.4(b).
\section*{Dystrybucja pracy}
Sekcję tą autor zaczyna od przedstawienia czym jest metoda dystrybucji pracy oraz podania przykładu jej wykorzystania (system DNS). Następnie autor pobieżnie opisuje działanie systemu DNS. Na koniec autor podaje i opisuje drugi przykład dystrybucji pracy(sieć internetowa World Wide Web).
\end{document}