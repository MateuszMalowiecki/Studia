
\documentclass[10pt, a4paper]{article}

\usepackage[polish]{babel}
\usepackage{polski}
\usepackage{iftex}
\ifTUTeX
  \usepackage{fontspec}
\else
  \usepackage[T1]{fontenc}
  \usepackage[utf8]{inputenc} % The default since 2018
  \DeclareUnicodeCharacter{200B}{{\hskip 0pt}}
\fi
\usepackage{graphicx}
\author{Mateusz Małowiecki}
\title{Opracowanie noty 2.6 Flooding versus random walks​ z rozdziału Architectures z książki MvS \& AST}
\date{17 marca 2021}

\begin{document}
%\maketitle
\section*{Wstęp}
Na pierwszy rzut oka, może się wydawać, że algorytm zalewania jest dużo lepszy niż algorytm losowych ścieżek, gdyż przeszukuje na raz więcej węzłów i jest w stanie szybciej znaleźć rozwiązanie. Jednakże w praktyce często mamy do czynienia ze zwielokrotnionymi danymi i badania pokazały, że nawet w przypadku, gdy współczynnik zwielokrotnienia jest niewielki, algorytm losowych ścieżek jest nie tylko efektywny, ale też bardziej wydajny w porównaniu z algorytmem zalewania​.
\section*{Obliczenia (algorytm losowych ścieżek)}
Załóżmy, że mamy N węzłów, każdy element danych znajduje się na r losowo wybranych maszynach. Poszukiwania polegają na losowym wybieraniu kolejnych węzłów, póki poszukiwany element nie zostanie znaleziony. Jeśli $P[k]$ jest prawdopodobieństwem znalezienia elementu po k próbach, to:​
\begin{equation}
P[k] = \frac{r}{N}*(1-\frac{r}{N})^{n-1}
\end{equation}
Niech $S$ będzie oczekiwaną liczbą węzłów, które musimy
odwiedzić przed znalezieniem żądanej pozycji danych. Wtedy:
\begin{equation}
S = \sum_{k=1}^{n} k * P[k] = \sum_{k=1}^{n} k *\frac{r}{N}*(1-\frac{r}{N})^{n-1}
\end{equation}
Stosując proste przekształcenia można oszacować, że $S \approx N/r $. Możemy zatem zauważyć, że jeśli $r = N$, to $S=1$, więc algorytm losowych ścieżek jest lepszy od algorytmu zalewania. Aczkolwiek gdy $r$ jest dużo mniejsze niż $N$ (przykładowo $r/N = 0.1\% $), to oczekiwana liczba węzłów wyniesie 1000.
\section*{Obliczenia (algorytm zalewania)}
W celu dokonania porównania między algorytmem losowych ścieżek, a algorytmem zalewania, załóżmy, że w algorytmie zalewania pierwszy wierzchołek wysyła wiadomość do $d$ wybranych sąsiadów, a każdy następny wierzchołek przesyła ją dalej do $d-1$ wybranych sąsiadów. Wówczas po $k$ krokach osiągniemy co najwyżej
\begin{equation}
R(k) = d*(d − 1)^{k − 1}
\end{equation}
węzłów. Zatem jeżeli wykonamy $k$ kroków (dla takiego $k$, że $\frac{r}{N}*R(k) \geq 1 $) to z dużym prawdopodobieństwem znajdziemy węzeł który zawiera poszukiwany element danych.
\section*{Porównanie}
Rozważmy jeszcze raz przypadek kiedy  $r/N = 0.1\%$ (czyli $S = 1000$). Jeśli w algorytmie zalewania przyjmiemy $d=10$ to po czterech krokach  osiągniemy 7290 węzłów, czyli zdecydowanie więcej niż 1000. Jednakże dla $d=33$, po zaledwie 2 krokach osiągniemy ok. 1000 węzłów i jednocześnie spełnimy warunek  $\frac{r}{N}*R(k) \geq 1 $. Oczywistą wadą algorytmu losowych ścieżek jest też to, że wdrażanie tego algorytmu może potrwać znacznie dłużej, zanim odpowiedź zostanie zwrócona.
\end{document}