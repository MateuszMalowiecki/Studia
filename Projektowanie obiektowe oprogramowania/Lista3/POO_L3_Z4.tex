\documentclass[10pt, a4paper]{article}

\usepackage[english]{babel}
\usepackage{polski}
\usepackage[utf8]{inputenc}
\usepackage{graphicx}
\usepackage{float}
\title{Rozwiązanie zadania 4 z zestawu 3 z "Projektowania obiektowego oprogramowania"}
\author{Mateusz Małowiecki}
\date{20 marca 2021}
\begin{document}
\maketitle
Zaprojektowano klasy Rectangle i Square i w
”naturalny” sposób relację dziedziczenia między nimi (każdy kwadrat jest prostokątem).
\begin{verbatim}
public class Rectangle {
    public virtual int Width { get; set; }
    public virtual int Height { get; set; }
}
public class Square : Rectangle {
    public override int Width {
        get { return base.Width; }
        set { base.Width = base.Height = value;}
    }
    public override int Height {
        get { return base.Height; }
        set { base.Width = base.Height = value; }
    }
}
\end{verbatim}
Co można powiedzieć o spełnianiu przez taką hierarchię zasady LSP w kontekście poniższego kodu klienckiego?
\begin{verbatim}
public class AreaCalculator {
    public int CalculateArea( Rectangle rect ) {
        return rect.Width * rect.Height;
    }
}
int w = 4, h = 5;
Rectangle rect = new Square() { Width = w, Height = h };
AreaCalculator calc = AreaCalculator();
Console.WriteLine( "prostokąt o wymiarach {0} na {1} ma pole {2}",
w, h, calc.CalculateArea( rect ) );
\end{verbatim}
Jak należałoby zmodyfikować przedstawioną hierarchię klas, żeby zachować zgodność z LSP
w kontekście takich wymagań? Jak potraktować klasy Rectangle i Square? Odpowiedź
zilustrować działającym kodem.
\section*{Rozwiązanie}
\subsection*{Co można powiedzieć o spełnianiu przez taką hierarchię zasady LSP w kontekście kodu klienckiego?}
Hierarchia ta łamie zasadę LSP. Powodem jest to, że prostokąty nie mogą być reprezentowane jako kwadraty, gdyż settery kwadratów zachowują się w niewłaściwy sposób (podstawienie Width = w, Height = h, może spowodować, że pole figury będzie inne niż w*h).
\subsection*{Jak należałoby zmodyfikować przedstawioną hierarchię klas, żeby zachować zgodność z LSP w kontekście takich wymagań? Jak potraktować klasy Rectangle i Square?}
Ponieważ Square nie rozszerza Rectangle to moglibyśmy usunąć relację dziedziczenia między Rectangle a Square. Dodatkowo ponieważ dla obu tych figur chcielibyśmy móc policzyć ich pola, to, żeby nie musieć obsługiwać osobno dwóch typów, moglibyśmy stworzyć abstrakcyjną klasę Shape, która będzie miała metodę GetArea, obliczającą pole danej figury. Takie rozwiązanie pozwala na zachowanie zasady LSP, ponieważ w dowolnym kontekście klasę Shape możemy zastąpić obiektami podklas Rectangle i Square.
\subsection*{Odpowiedź zilustrować działającym kodem}
Kod znajduje się w plikach "POO\_L3\_Z4\_After.cs" i "POO\_L3\_Z4\_Before.cs". W pliku "POO\_L3\_Z4\_Test.cs" znajdują się testy.
\end{document}