\documentclass[10pt, a4paper]{article}

\usepackage[english]{babel}
\usepackage{polski}
\usepackage[utf8]{inputenc}
\usepackage{graphicx}
\title{Rozwiązanie zadania 2 z zestawu 1 z "Projektowania obiektowego oprogramowania"}
\author{Mateusz Małowiecki}

\begin{document}
\maketitle
\section*{Polecenie}
Zdokumentować dwa przypadki użycia wybranego przez siebie przykładowego problemu (gra w brydża, zakupy w sklepie internetowym, inne). Co najmniej jeden opisać w
formie skróconej (brief) i jeden w formie pełnej (fully dressed).
\section*{Rozwiązanie}
\subsection*{Wybrany problem}
Użycie konta
\subsection*{Przypadek 1(forma skrócona)}
\textit{Wypłać pieniądze} - klient staje przed bankomatem, a następnie wkłada kartę do czytnika. Bankomat pyta klienta o język interfejsu. Klient wybiera język. Bankomat prosi o podanie numeru PIN. Klient wpisuje numer PIN. Bankomat weryfikuje numer PIN i podaje możliwe opcje. Klient wybiera opcję wypłaty gotówki. Bankomat prosi klienta o podanie kwoty. Klient wpisuje kwotę. Bankomat dokonuje transakcji. Bankomat wysuwa kartę i prosi klienta o jej zabranie. Klient zabiera kartę. Bankomat wydaje kwotę, którą podał klient i prosi o zabranie pieniędzy. Klient zabiera pieniądze. 
\subsection*{Przypadek 2(forma pełna)}
\textit{Sprawdzanie stanu konta}
\subsubsection*{Nazwa:}
Sprawdzanie stanu konta
\subsubsection*{Poziom ważności:}
Wysoki
\subsubsection*{Aktor główny:} 
Klient
\subsubsection*{Interesariusze i ich cele:}
Klient banku - chce sprawdzić stan konta \\
Bankomat - chce zapewnić, że operacji na koncie dokonuje autoryzowany użytkownik \\
Bank: chce satysfakcji klienta. \\
\subsubsection*{Warunki początkowe}
Klient stoi przed bankomatem. Bankomat jest czynny.
\subsubsection*{Warunki końcowe}
Uwierzytelnienie kończy się sukcesem. Użytkownik wie jaką kwotę ma na koncie.
\subsubsection*{Główny scenariusz sukcesu}
1. Klient staje przed bankomatem. \\
2. Klient wkłada kartę do czytnika. \\
3. Bankomat pyta klienta o język interfejsu. \\ 
4. Klient wybiera język. \\
5. Bankomat prosi o podanie numeru PIN. \\
6. Klient wpisuje numer PIN. \\
7. Bankomat weryfikuje poprawność numeru PIN. \\ 
8. Bankomat podaje możliwe opcje. \\
9. Klient wybiera opcję sprawdzenia stanu konta. \\ 
10. Bankomat podaje kwotę znajdującą się na koncie klienta. \\
11. Klient kończy sprawdzanie stanu konta. \\
12. Bankomat wysuwa kartę i prosi klienta o jej zabranie. \\ 
13. Klient zabiera kartę. \\
\subsubsection*{Rozszerzenia}
2a. Klient wkłada kartę złą stroną: \\ 2a.1. bankomat wysuwa kartę i wyświetla informację, że karta została włożona złą stroną. \\ \\
2b. Bankomat nie rozpoznał karty: \\ 2b.1. Bankomat wysuwa kartę i wyświetla informację, że karta nie została rozpoznana. \\ \\
4a. Klient wybrał zły język interfejsu: \\4a.1. Klient wraca do ekranu z wyborem języka, za pomocą przycisku w lewym dolnym rogu ekranu. \\ 4a.2. Klient wybiera poprawny język \\ \\
7a. Klient podał błędny numer PIN: \\ 7a.1. Bankomat prosi klienta o ponowną próbę \\ 7a.2. Klient wpisuje PIN ponownie  \\ 7a.2a. Klient podał błędny numer PIN trzykrotnie: \\ 7a.2a.1. Bankomat wysuwa kartę \\ 7a.2a.2. Bankomat informuje klienta o zablokowaniu karty i konieczności skontaktowania się z bankiem. \\
\subsubsection*{Dodatkowe wymagania}
- Możliwość wyboru polskiego i angielskiego języka interfejsu \\
- Możliwość wypłacenia 100, 200, 500, 1000, 2000 lub 3000 PLN \\
- Czas uwierzytelnienia nie dłuższy niż 10 sekund
\subsection*{Technologia}
2 - czytnik karty \\
5 - Klawiatura do wprowadzenia numeru PIN \\
3, 9 - Przyciski do wyboru opcji w menu \\
\end{document}